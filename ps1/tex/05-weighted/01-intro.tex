\item \subquestionpoints{10} Consider a linear regression problem in which
we want to ``weight'' different training examples differently.  Specifically,
suppose we want to minimize
%
\begin{equation*}
	J(\theta) = \frac{1}{2} \sum_{i=1}^m w^{(i)}
		\left(\theta^Tx^{(i)} - y^{(i)}\right)^2.
\end{equation*}
%
In class, we worked out what happens for the case where all the weights (the
$w^{(i)}$'s) are the same. In this problem, we will generalize some of those
ideas to the weighted setting.
\begin{enumerate}
	\item \subquestionpoints{2} Show that $J(\theta)$ can also be written
    %
    \begin{equation*}
    J(\theta) = (X\theta - {y})^T W (X\theta - {y})
    \end{equation*}
    %
    for an appropriate matrix $W$, and where $X$ and ${y}$ are as
    defined in class. Clearly specify the value of each element of the matrix
    $W$.

	\item \label{item:lwr-solution} \subquestionpoints{4} If all the $w^{(i)}$'s
    equal 1, then we saw in class that the normal equation is
    %
    \begin{equation*}
    X^TX\theta = X^T{y},
    \end{equation*}
    %
	and that the value of $\theta$ that minimizes $J(\theta)$ is given by
	$(X^TX)^{-1}X^T{y}.$
	By finding the derivative $\nabla_\theta J(\theta)$ and setting that to zero,
	generalize
	the normal equation to this weighted setting, and give the new value of
	$\theta$ that minimizes
    $J(\theta)$ in closed form as a function of $X$, $W$ and ${y}$.

	\item \subquestionpoints{4} Suppose we have a dataset
	$\{(x^{(i)}, y^{(i)});\, i=1\ldots,m\}$ of $m$ independent examples, but
    we model the $y^{(i)}$'s as drawn from conditional distributions with
    different levels of variance $(\sigma^{(i)})^2$. Specifically, assume the
    model
    %
    \begin{equation*}
		p(y^{(i)} | x^{(i)} ; \theta) = \frac{1}{\sqrt{2\pi}\sigma^{(i)}} \exp\left(-
		\frac{(y^{(i)} - \theta^Tx^{(i)})^2}{2(\sigma^{(i)})^2}\right)
	\end{equation*}
    %
    That is, each $y^{(i)}$ is drawn from a Gaussian distribution with mean
    $\theta^Tx^{(i)}$ and variance $(\sigma^{(i)})^2$ (where the
    $\sigma^{(i)}$'s are fixed, known, constants). Show that finding the
    maximum likelihood estimate of $\theta$ reduces to solving a weighted
	linear regression problem.  State clearly what the $w^{(i)}$'s are in terms of
    the $\sigma^{(i)}$'s.
\end{enumerate}

\ifnum\solutions=1
  \begin{answer}\\
For 5a if we let $W$ be the diagonal matrix\\
\[
\begin{bmatrix}
\frac{1}{2}w^{(1)} & 0 & \dots & 0 \\
0 & \frac{1}{2}w^{(2)} & \dots & 0 \\
\vdots & \vdots & \vdots  & \vdots \\
0 & 0 & \dots & \frac{1}{2}w^{(m)} \\
\end{bmatrix}\\
\]
Then we can express $J(\theta)$ as \\
$J(\theta)=(X \theta - y)^TW(X \theta -y)$\\
Note that we could remove the $\frac{1}{2}$ from the weight vector $W$ and write it as $J(\theta)=\frac{1}{2}(X \theta - y)^TW(X \theta -y)$ \\\\\\
For 5b we know that $J(\theta)=(X\theta-y)^TW(X\theta-y)$\\
$=\theta^TX^TWX\theta -\theta^TX^TWy-y^TWX\theta+y^TWy$\\
Since this is a scalar, and $tr(a)=a$ is $a$ is a scalar, therefore we have\\
$J(\theta)=tr(\theta^TX^TWX\theta -\theta^TX^TWy-y^TWX\theta+y^TWy)=tr(\theta^TX^TWX\theta) -tr(\theta^TX^TWy)-tr(y^TWX\theta)+tr(y^TWy)$\\
Then $\nabla_{\theta}J(\theta)=\nabla_{\theta}tr(\theta^TX^TWX\theta) -\nabla_{\theta}tr(\theta^TX^TWy)-\nabla_{\theta}tr(y^TWX\theta)+\nabla_{\theta}tr(y^TWy)$\\
$=(X^TWX)^T\theta+X^TWX\theta-2(Y^TWX)^T$\\
$=2X^TWX\theta-2X^TWy$\\\\
Setting this to zero, we have
$X^TWX\theta=X^TWy \implies \theta=(X^TWX)^{-1}X^TWy$\\
\\
\\
For 5c $p(y^{(i)}|x^{(i)},\theta)=\frac{1}{\sqrt{2 \pi} \sigma^{(i)}} e^-\frac{(y^{(i)}-\theta^Tx^{(i)})^2}{2(\sigma^{(i)})^2}=\frac{1}{\sqrt{2 \pi} \sigma^{(i)}} e^-\frac{(\theta^Tx^{(i)}-y^{(i)})^2}{2(\sigma^{(i)})^2}$\\
$\therefore L(\theta)=\prod_{i=1}^{m}p(y^{(i)}|x^{(i)},\theta)=\prod_{i=1}^{m}\frac{1}{\sqrt{2 \pi} \sigma^{(i)}} e^-\frac{(\theta^Tx^{(i)}-y^{(i)})^2}{2(\sigma^{(i)})^2}$\\
$\therefore l(\theta)=log L(\theta)=\sum_{i=1}^{m}log\frac{1}{\sqrt{2 \pi}\sigma^{(i)}} -\frac{1}{2}\sum_{i=1}^{m}\frac{(\theta^T x^{(i)}-y^{(i)})^2}{(\sigma^{(i)})^2}$\\
Maximizing $l(\theta)$ is the same as minimizing $\frac{1}{2}\sum_{i=1}^{m}\frac{(\theta^T x^{(i)}-y^{(i)})^2}{(\sigma^{(i)})^2}$\\
$=\frac{1}{2}\sum_{i=1}^{m}\frac{1}{(\sigma^{(i)})^2} (\theta^T x^{(i)}-y^{(i)})^2$\\
This reduces to a weighted linear regression where $w^{(i)}=\frac{1}{(\sigma^{(i)})^2}$
\end{answer}

\fi
