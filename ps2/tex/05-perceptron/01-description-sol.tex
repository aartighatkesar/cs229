\begin{answer}\\
i.\\
We know that\\
$\theta^{(0)}=0$\\
$\theta^{(1)}=\alpha(y^{(1)}-h_{\theta^{(0)}}(\phi(x^{(1)}))\phi(x^{(1)})$\\
We can represent $\alpha(y^{(1)}-h_{\theta^{(0)}}(\phi(x^{(1)}))$ as $\beta_1$\\
$\therefore \theta^{(1)}=\beta_1 \phi(x^{(1)})$\\
$\theta^{(2)}=\theta^{(1)}+\alpha(y^{(2)}-h_{\theta^{(1)}}(\phi(x^{(2)}))\phi(x^{(2)})$\\
Again we can represent $\alpha(y^{(2)}-h_{\theta^{(1)}}(\phi(x^{(2)}))$ as $\beta_2$\\
$\therefore \theta^{(2)}=\beta_1 \phi(x^{(1)})+\beta_2 \phi(x^{(2)})$\\
$\theta^{(3)}=\theta^{(2)}+\alpha(y^{(3)}-h_{\theta^{(2)}}(\phi(x^{(3)}))\phi(x^{(3)})$\\
similarly we can represent $\alpha(y^{(3)}-h_{\theta^{(2)}}(\phi(x^{(3)}))$ as $\beta_3$\\
$\therefore \theta^{(3)}=\beta_1 \phi(x^{(1)})+\beta_2 \phi(x^{(2)})+\beta_3 \phi(x^{(3)})$\\
In general $\theta^{(i)}=\sum_{j=1}^{i}\beta_j \phi(x^{(j)})$\\
$\theta_0$ is initialized with no coefficients (equivalent to zero).\\
\\
\\
ii.\\
We showed above that $\theta^{(i)}=\sum_{j=1}^{i}\beta_j \phi(x^{(j)})$\\
therefore to calculate $h_{\theta^{(i)}}(X^{(i+1)})$ we just do it as $\sum_{j=1}^{i}\beta_j K(\phi(x^{j}),\phi(x^{(i)}))$\\
This is more efficient than working in the high dimension space.\\
\\
\\
iii. \\
Since $(y^{(1)}-h_{\theta^{(0)}}(\phi(x^{(1)})) \neq 0$ only on misclassified examples, $\theta$ is only updated on these examples. So we dont need to store the ones that are classified correctly since they dont update $\theta$. This is similar to above, but slightly more efficent since we have a smaller list of examples.
\end{answer}
