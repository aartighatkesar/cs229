\clearpage
\item \points{18} {\bf Constructing kernels}

In class, we saw that by choosing a kernel $K(x,z) = \phi(x)^T\phi(z)$, we can
implicitly map data to a high dimensional space, and have the SVM algorithm
work in that space.  One way to generate kernels is to explicitly define the
mapping $\phi$ to a higher dimensional space, and then work out the
corresponding $K$.

However in this question we are interested in direct construction of kernels. 
I.e., suppose we have a function $K(x,z)$ that we think gives an appropriate
similarity measure for our learning problem, and we are considering plugging
$K$ into the SVM as the kernel function. However for $K(x,z)$ to be a valid
kernel, it must correspond to an inner product in some higher dimensional space
resulting from some feature mapping $\phi$.  Mercer's theorem tells us that
$K(x,z)$ is a (Mercer) kernel if and only if for any finite set $\{x^{(1)},
\ldots, x^{(m)}\}$, the square matrix $K \in \Re^{m \times m}$ whose entries
are given by $K_{ij} = K(x^{(i)},x^{(j)})$ is symmetric and positive
semidefinite. You can find more details about Mercer's theorem in the notes,
though the description above is sufficient for this problem.

Now here comes the question: Let $K_1$, $K_2$ be kernels over $\Re^n \times
\Re^n$, let $a \in \Re^+$ be a positive real number, let $f : \Re^n \mapsto
\Re$ be a real-valued function, let $\phi: \Re^n \rightarrow \Re^d$ be a
function mapping from $\Re^n$ to $\Re^d$, let $K_3$ be a kernel over $\Re^d
\times \Re^d$, and let $p(x)$ a polynomial over $x$ with \emph{positive}
coefficients.

For each of the functions $K$ below, state whether it is necessarily a
kernel.  If you think it is, prove it; if you think it isn't, give a
counter-example.

\begin{enumerate}

\item \subquestionpoints{1} $K(x,z) = K_1(x,z) + K_2(x,z)$
\item \subquestionpoints{1} $K(x,z) = K_1(x,z) - K_2(x,z)$
\item \subquestionpoints{1} $K(x,z) = a K_1(x,z)$
\item \subquestionpoints{1} $K(x,z) = -a K_1(x,z)$
\item \subquestionpoints{5} $K(x,z) = K_1(x,z)K_2(x,z)$ 
\item \subquestionpoints{3} $K(x,z) = f(x)f(z)$
\item \subquestionpoints{3} $K(x,z) = K_3(\phi(x),\phi(z))$
\item \subquestionpoints{3} $K(x,z) = p(K_1(x,z))$

\end{enumerate}

[\textbf{Hint:} For part (e), the answer is that $K$ \emph{is} indeed
a kernel. You still have to prove it, though.  (This one may be harder than the
rest.)  This result may also be useful for another part of the problem.]

\ifnum\solutions=1 {
  \begin{answer}\\
(a). We know that $K_1$ and $K_2$ are kernels. Therefore by definition they are positive semi definite.\\
Therefore $\forall z$, $z^TK_1z \geq 0$ and $\forall z, z^TK_2z \geq 0$\\
$\therefore z^TKz = z^TK_1z+z^TK_2z \geq 0$. Therefore $K$ is positive semidifinite. Therefore $K$ is a kernel.\\
\\
(b) From part (c) below we know that a positive number $\times$ a Kernel is a Kernel. So if we take $K2=aK_1 (a \geq 2)$ then $K=K_1-aK_1=(1-a)K_1$\\
But $1-a <0$ therefore using proof in part (d) we can say it is not a Kernel.\\
(c) $\forall z$, $z^TKz=z^TaK_1z=az^TK_1z=az^TK_1z$\\
Since $K_1$ is a Kernel, $z^TK_1z \geq 0$. Also since $a$ is a positive real number, $az^TK_1z \geq 0$ therefore it is a Kernel.\\
(d) $\forall z$, $z^TKz=z^T(-aK_1)z=-az^TK_1z$\\
Since $K_1$ is a Kernel, $z^TK_1z \geq 0$. Also since $a$ is a positive real number, $-az^TK_1z \leq 0$ therefore it is NOT a Kernel.\\
(e) We know that $K_1$ and $K_2$ are PSD\\
Therefore $z^TK_1z \geq 0$ and $z^TK_2z \geq 0$\\
Also $Z=zz^T$ is PSD. Let $U$ be orthogonal matrix. Then $UU^T=U^TU=I$\\
Then we know that $Z=U\Lambda U^T$\\
And $(U\Lambda U^T)^{-1}=(U^T)^{-1}(U \Lambda)^{-1}=U \Lambda^{-1} U^T$\\
$\therefore (zz^T)(zz^T)^{-1}(zz^T)^{-1}(zz^T)=I$\\
$\implies zz^T U \Lambda^{-1} U^T U \Lambda^{-1} U^T zz^T=I$\\
$\implies zz^T U \Lambda^{-1} \Lambda^{-1} U^T zz^T=I$\\
$\implies zz^T U \Lambda^{-2} U^T zz^T=I$\\
Let $x=U^Tz$. then $x^T=z^TU$. Therefore we can rewrite the above as\\
$z x^T \Lambda^{-2} x z^T=I$\\
Finally we have \\
$z^TKz=z^TK_1K_2z=z^TK_1IK_2z=z^TK_1 z x^T \Lambda^{-2} x z^TK_2z$\\
Since $\Lambda^{-2}$ is a diagonal matrix, it is positive semidefinite. Therefore $x^T \Lambda^{-2} x \geq 0$ and we already know that $z^TK_1 z \geq 0$ and $z^TK_2 z \geq 0$\\
Therefore the RHS is $\geq 0$\\
Therefore $z^TKz \geq 0$ and therefore it is a Kernel\\ 
(f) $K(x,y)=f(x)f(z)$.\\
Then $z^TKz=z_1^2f(x^{(1)})^2+2z_1z_2f(x^{(1)})f(x^{(2)})+z_2^2f(x^{(2)})^2+2z_2z_3f(x^{(2)})f(x^{(3)})+z_3^2f(x^{(3)})^2 \dots +z_3^2f(x^{(3)})^2$\\
$=(\sum_{i=1}^{m}z_i f(x^{(i)}))^2 \geq 0$\\
Therefore it is a kernel\\
(g) Since $K=K_3(\phi(x),\phi(z))$ it is Symmetric\\
Also $z^TKz=z^TK_3z \geq 0$ so K is a kernel.
(h) The polynomial can be written as $a_tK_3(x,z)^t+a_{t-1}K_3(x,z)^{t-1}+a_{t-2}K_3(x,z)^{t-2}+\dots+a_1K_3(x,z)+a_0$\\
From (e) the power of a Kernel is a Kernel\\
From (c) a positive number times a Kernel is a Kernel\\
Therefore each of the terms of the polynomial is a Kernel\\
Finally from (a) the sum of 2 kernels is a kernel. Therefore the polynomial is a kernel.
\end{answer}
} \fi
